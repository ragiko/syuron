\chapter{言語的特徴量}

\section{a}

% 前後が役に立つことは分かっている。南条
% 金沢の人文献
% 周囲の言語を含めた特徴量を含めた

% TODO: 参考文献 音響学会15 金寺
NTCIR11 SpokenQuery\&Doc Formal-run の SQSCR SGS retrieval条件では,検索対象の文書に対し,周囲の文書を利用すると検索精度が向上する事が報告されている.金寺ら[]は音声内容検索において,シーンべクトルの線形補間を用いて音
声情報検索を行った結果,通常の質問文によ
る結果と同様にシーンべクトルの線形補間の
検索性能が高いことが報告されている. \\
そこで本研究では,クエリ尤度を用いた検索モデルに対し,検索対象の周囲の単語を考慮することで,検索精度を改善する事を考案する.


% クエリ尤度と拡張
% バイグラム拡張例

% 言語モデル的に使えそうな特徴 molkov
% cache
% RNN
% KN

% 参考: http://www.ar.media.kyoto-u.ac.jp/lab/bib/report/NEM-sdp07.pdf
\subsection{キャッシュモデルに基づく言語モデルの適応}
キャッシュモデルでは,単語 $w_n$ の直前の単語履歴をキャッシュ $H = \{ w_{n-|H|}, ..., w_{n-1}\} $ として記憶し,これに含まれる単語が再び使用される確率が高いと予測する.このキャッシュに基づく単語 $w_n$ の出現確率 $P_c(w_n|H)$ は式(\ref{cache})によって与えられる. ただし,$|H|$ は単語履歴 $H$ の長さ, $\delta$ はクロネッカーのデルタである.

\begin{equation}
		P_c(w_n|H) = \frac{1}{|H|} \sum_{w_h \in H} \delta (w_n, w_h)
    \label{cache}
\end{equation}

% 参考: http://www.cl.cs.titech.ac.jp/~fujii/paper/asj2002akiba.pdf
\subsection{N-gramとスムージング}

N-gram言語モデルでは、学習データに現れない単語列を扱うため、種々の平滑化(スムージング)手法が適用される。スムージングの一つとして,バックオフ・スムージングでは、高次のN-gramが存在しない場合、低次のN-gramで代用する。

\subsection{単語連接を重視したバックオフ平滑化手法}
バックオフ・スムージングの一般式は次のように表される。

\begin{equation}
		P(w_i|w_{i-n+1}^{i-1}) = 
    \begin{cases} 
        d_{w_{i-n+1}^i} P_{ML}(w_i|w_{i-n+1}^{i-1}) & C(w_{i-n+1}^{i-1}) > 0\\ 
        \alpha(w_{i-n+1}^{i-1})P(w_i|w_{i-n+2}^{i-1}) & C(w_{i-n+1}^{i-1}) = 0
    \end{cases} 
    \label{ngram_smoosing1}
\end{equation}

ここで $d$, $P_{ML}$, $\alpha$は、それぞれ、ディスカウント係
数、最尤推定によるN-gram確率、確率の総和を1とするための正規化係数である。

\subsection{Kneser-Ney スムージング}
KneserとNey[3]は、絶対法[4]を拡張した平滑化手法を示している
純粋に平滑化手法として他の手法と比べた場合でも、英語に適用した例で優れた性能を示すことが報告されている[2]。Kneser-Neyスムージングでは、高次のN-gram確率が利用できない(信頼できない)場合に使用する低次の確率として、最尤推定による確率 $P_{ML} (w_i|w_{i-n+1}^{i-1})$ の代わりに次の値 $P_{KN} (w_i|w_{i-n+1}^{i-1})$ を用いる.

\begin{equation}
		P_{KN} (w_i|w_{i-n+1}) = \frac{|\{w_{i-n}|C(w_{i-n+1}^{i-1}) > 0\}|}{\sum_{w_i} |\{w_{i-n}|C(w_{i-n+1}^{i-1}) > 0\}|} 
    \label{ngram_smoosing2}
\end{equation}

Kneser-Neyスムージングでは、長さ $N$ のN-gramについて, $n  <  N$ である全ての $n$ に対し,$P_{KN} (w_i|w_{i-n+1}^{i-1})$ を用いる。
