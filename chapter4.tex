%%%%%%%%%%%
% chapter2からのコピペ
%
\chapter{音声ドキュメント中の認識誤り単語の棄却手法}

音声認識結果には音声認識誤りが含まれる.これらの認識誤りは,その後の特徴量計算の段階で悪影響を及ぼし,
最終的に文書検索精度を劣化させる原因となる.
本節では音声認識結果から,単語の文脈一貫性の観点から認識誤り単語を推定し,棄却する手法を述べる.
%
% TODO:どこかにベクトル空間モデルに基づく文書比較についての話を

% TODO:卒論からのコピーなので,一部咬み合わない,上と重複する部分がある.

\section{単語の文脈一貫性に基づく誤り単語の棄却}  \label{sec_word_rejection}
% TODO:誤り単語が周辺と脈絡なく...要出典,要図
音声認識誤り単語は一般にその周辺の単語と脈絡なく現れ,関連が弱いという特徴がある.
そこで本研究では前述のPMIを用いて,単語と文書の関連度を推定し,関連が弱い語を認識誤り単語として棄却する枠組みを提案する.
文書$d = {w_1, w_2, ..., w_i, ..., w_n}$と,$d$中の単語$w_i$の関連度を,式(\ref{eq_sumpmi})で計算される$sumPMI(w_i)$で推定する.
\begin{equation}
    sumPMI(w_i) = \sum^{n}_{j=1, j \neq i}{PMI(w_i, w_j)}   \label{eq_sumpmi}
\end{equation}
PMIは2単語間の関連の強さを表すため,$sumPMI(w_i)$は単語$w_i$と文書$d$の関連の強さを表すと考えられる.
$sumPMI$が低い語は,文書との関連が弱い語であり,認識誤り単語であることが期待できる.
本研究では,$sumPMI(w)$の値が負となる単語$w$を,認識誤り単語として除外した.

