%%%%%%%%%%%%%%%%%%%%%%%%%%%%%%%%%%%%%%%%%%%%%
% 第1章分割

\chapter{序論} % 章の見出し
\section{背景}
近年,講演の録音やニュース番組,動画などの音声情報を含むコンテンツが増加してきている.
このようなコンテンツを検索する際には通常,予め人手でタイトルや概要などのメタ情報を付与しておき,そのメタ情報を基にしてユーザの求める情報かどうかを判定し,検索結果を出力する.
一方でそのような検索手法は,人手でメタ情報を付与する必要があるため,人的コストがかかるため,音声を音声認識によってテキストの形に起こし,テキスト検索技術を用いて検索する手法が提案されている.

このテキスト検索技術には,音声認識結果を利用するため,一般のテキスト検索手法とは異なる方法を用いる.例えば,音声認識結果には音声認識誤りが存在し,これらの認識誤りはその後の特徴量計算の段階で悪影響を及ぼたり,音声は基本的に話し言葉で録音されるため,冗長な話し言葉の中から複数のキーワードを認識し抽出しなければならないため,それらを考慮したテキスト検索手法が必要とされる.そのため,音声認識結果を利用したテキスト検索技術に対し,多くの研究が行なわれている. 

\subsection{NTCIR}
% TODO: 参考文献
NTCIR(NII Test Collection for Information Retrieval)\cite{NTCIR}とは,膨大な情報の中から所望の情報にアクセスする技術の発展を図るプロジェクトである.
NTCIRでは研究分野に応じて情報検索,質問応答,要約,テキストマイニング,機械翻訳など,様々なタスクが用意される.
各タスクでは,評価の為のテストコレクションと,その実験結果を評価するためのツールが用意される.
タスクの参加者は,共通のテストコレクションの上で研究し,精度を競い合う.
NTCIRの目的には,次の3つが挙げられる.
\begin{enumerate}
    \item 大規模かつ再利用可能なテストコレクションの構築
    \item 多数の研究者が研究成果を提供し,相互に学び合うフォーラムの形成
    \item 情報アクセス技術と,その評価手法,評価指標に関する研究の推進
\end{enumerate}

\subsection{SpokenQuery\&Doc}
SpokenQuery\&Docでは,大量に存在する音声データからユーザに有用性のあるデータを検索する事を目的とする.本タスクでは,2種類のサブタスクから音声文書の検索を目指す.STD(Spoken Term Detection)サブタスクでは単語クエリが与えられ,大量に存在する音声文書の中から単語クエリが発話されている文書の部位を検索し,検索精度と検索にかかった時間で評価する.一方SDR(Spoken Document Retrieval)サブタスクでは質問文形式の質問クエリが与えられ,検索対象の音声文書から質問クエリの内容に該当する文書(講演単位:Lecture retrieval)又は文書中の部位(部位単位:Passage retrieval)を検索する.筆者らはSpokenDoc SDR Passage retrievalに参加しており,本紙における音声文書検索問題とは当タスクの問題を指すこととする.

\section{研究目的}
前節で解説したSpokenDocタスクのように,音声認識結果と質問クエリが存在する場合,質問クエリを文書として見ることで,音声認識結果テキストとの類似度を検索に用いる手法が考えられる.

この検索方法において,まず音声認識精度を改善し,認識誤りが減少すれば,テキスト検索手法の精度を改善できると考えられる.そのため,本研究では精度の高い音声認識エンジンを利用し,検索精度を向上させる.

また,類似度計算の特徴量としては一般的にTF-IDF法が用いられるが,今回のSDRタスクのように質問クエリのテキストの長さが不十分な場合,TF-IDF法ではうまく推定を行う事ができない.またTF-IDF法は音声認識誤りや言い換え問題に対しても弱いため,そのまま文書検索に用いるには疑問が残る.
本研究はこれらの問題の解決を試みることで,音声文書の検索の精度向上を図る.まずTF-IDF法による文書間の類似度を用いない手法,すなわち確率的言語モデルを用いた音声文書の検索について述べる.また,確率的言語モデルとして一般的であるクエリ尤度モデル \cite{query_likelihood}についてクエリ拡張を行い,その有用性について検討する.

今回のSDRタスクにおいて,検索対象となる文書のテキストの長さが短い場合が存在し,この場合テキストの情報が少ないため,テキスト検索精度が低い場合がある.そのため,本研究では検索対象となる文書の周囲のテキストを利用し,検索精度の向上を図る.
周囲のテキストを利用する具体的な方法としては,キャッシュモデル・N-gram言語モデル・リカレントニューラルネットワーク言語モデルなどの,単語が出現する順序に着目した言語的特徴量を利用したり,検索対象となる文書の周囲の文書と質問クエリとの類似度スコアを統合したりする.これらの手法による音声文書の検索について考察を行う.

\section{論文構成}
\noindent
本論文の構成を以下に示す.

第2章では,音声文書の検索の基礎技術として,TF-IDF法をはじめとした自然言語処理技術について述べる.

第3章では,音声認識精度と検索精度について述べる.音声認識精度の高い認識エンジンが生成した音声文書が検索精度にどのように影響するのかを分析する 

第4章では,クエリの関連文書を用いたクエリ尤度モデルの拡張について説明する.Web文書と論文を用いて,クエリ尤度モデルを拡張する.

第5章では,文脈を考慮した言語特徴量の検討について説明する.言語特徴量としてよく近年利用されるキャッシュモデル,N-gram言語モデル,リカレントニューラルネットワーク言語モデルについて,それぞれ評価し,影響を分析する.

第6章では,前後の部分文書の情報の加味した時の検索精度の改善について述べる. 

そして,第7章では提案した手法において,有効な方法を統合し,更なる検索精度の改善を目指す. 

最後に,第8章でまとめと今後の課題を述べる.

% TODO: 
% 章が確定してないため,確定次第書きます。流れとしては
% ▶︎クエリ尤度に対して、Web文書と論文文書の統合
% ▶︎クエリ尤度,クエリ尤度 + キャッシュ, クエリ尤度 + RNN のMAP値比較
% ▶︎それらの手法 + スライド統合でのMAP値
% 論文構成を以下に示す.
