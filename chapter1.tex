%%%%%%%%%%%%%%%%%%%%%%%%%%%%%%%%%%%%%%%%%%%%%
% 第1章分割

\chapter{序論} % 章の見出し
\section{背景}
近年,講演の録音やニュース番組,動画などの音声情報を含むコンテンツが増加してきている.
このようなコンテンツを検索する際には通常,予め人手でタイトルや概要などのメタ情報を付与しておき,そのメタ情報を基にしてユーザの求める情報かどうかを判定し,検索結果を出力する.
一方でそのような検索手法は,人手でメタ情報を付与する必要があるため,人的コストがかかるため,音声を音声認識によってテキストの形に起こし,テキスト検索技術を用いて検索する手法が提案されている.


\subsection{NTCIR}
NTCIR(NII Test Collection for Information Retrieval)\cite{NTCIR}とは,膨大な情報の中から所望の情報にアクセスする技術の発展を図るプロジェクトである.
NTCIRでは研究分野に応じて情報検索,質問応答,要約,テキストマイニング,機械翻訳など,様々なタスクが用意される.
各タスクでは,評価の為のテストコレクションと,その実験結果を評価するためのツールが用意される.
タスクの参加者は,共通のテストコレクションの上で研究し,精度を競い合う.
NTCIRの目的には,次の3つが挙げられる.
\begin{enumerate}
    \item 大規模かつ再利用可能なテストコレクションの構築
    \item 多数の研究者が研究成果を提供し,相互に学び合うフォーラムの形成
    \item 情報アクセス技術と,その評価手法,評価指標に関する研究の推進
\end{enumerate}

\subsection{SpokenQuery\&Doc}
SpokenQuery\&Docでは,大量に存在する音声データからユーザに有用性のあるデータを検索する事を目的とする.本タスクでは,2種類のサブタスクから音声文書の検索を目指す.STD(Spoken Term Detection)サブタスクでは単語クエリが与えられ,大量に存在する音声文書の中から単語クエリが発話されている文書の部位を検索し,検索精度と検索にかかった時間で評価する.一方SDR(Spoken Document Retrieval)サブタスクでは質問文形式の質問クエリが与えられ,検索対象の音声文書から質問クエリの内容に該当する文書(講演単位:Lecture retrieval)又は文書中の部位(部位単位:Passage retrieval)を検索する.筆者らはSpokenDoc SDR Passage retrievalに参加しており,本紙における音声文書検索問題とは当タスクの問題を指すこととする.

\section{研究目的}
前節のSpokenQuery\&Docにおいて,最も良い検索精度を示した検索手法は,TF-IDFをベースにしたものであった.一方でTF-IDFはクエリと文書の両方に存在する語でしか類似度を計算することが出来ない欠点を持つ.一般にクエリは小語彙であるため,文書検索には不向きである.またTF-IDFは音声文書検索において,一部の認識誤り単語に対して過剰に高い値を与えてしまう問題が存在する.そのため,TF-IDFと比較して,もっと有用な文書特徴量を調査すれば,検索精度を向上させることができると考えられる.
 そこで本研究では文書検索において,精度向上の要因を調査・分析し,考察した.

\section{論文構成}
\noindent
本論文の構成を以下に示す.\\
% TODO: 
% 章が確定してないため,確定次第書きます。流れとしては
% ▶︎クエリ尤度に対して、Web文書と論文文書の統合
% ▶︎クエリ尤度,クエリ尤度 + キャッシュ, クエリ尤度 + RNN のMAP値比較
% ▶︎それらの手法 + スライド統合でのMAP値
% 論文構成を以下に示す.
