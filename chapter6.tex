%%%%%%%%%%%%%%%%%%%%%%%%%%%%%%%%%%%%%%%%%%%%%
% 第7章分割

\chapter{結論} 
\section{まとめ}
本研究では,文書検索において検索精度に影響を及ぼす要因を4つ取りあげ,それらについて分析を行った.

まず,音声認識精度と検索精度に関しては,従来から使用しているJuliusによる音声認識結果より,高性能な音声認識システムであるKaldiを用いた音声認識結果を用いたときの方が,検索精度が上昇する事が確認できた.

クエリの関連文書を用いたクエリ尤度モデルの拡張では,論文を用いたときに,MAP値の向上が見られた.しかし,Web文書を用いたときは,MAP値にあまり変化が見られなかった.これは,Web文書がNTCIR11のクエリに対して,スムージングを行なえなかったためであると推察できる.論文を用いると,専門単語も広く言語モデルとして含められるため,スムージングできる可能性が高くなり,MAP値が上昇したのだと考えられる.

次に,前後の部分文書の情報の加味では,精度を大幅に向上することを確認した.今回のSpokenQuery\&Docは,1つのクエリに対する正解文書が,1箇所に固まっていることが多かったため,特に効果が大きかったと考えられる.

最後に,これまでに説明してきた手法の中で,検索性能の向上に有効な結果を示したものを統合し,検索精度の更なる向上を図った.その結果から,前後の部分文書の情報を加味した場合,MAP値は大幅に改善した.しかし,キャッシュモデルに前後の部分文書の情報を加味した場合,特徴量が重複してしまい,良い結果が得られなかった.

\section{今後の課題}
今後の課題としてまず,ディリクレスムージング時のパラメータの設定があげられる.今回は人手で調整し,MAP値が高くなる値を設定したが,今後は尤度関数を設定し,最適化することによりパラメータを調節したい.

TODO: その他, RNNとかKNが上手く行ってないところ,論文の精度が向上しないところ、キャッシュモデルとスライド統合が競合してしまっているところ

% 単語の意味的情報を,検索精度が向上するように従来の検索法に組み合わせることが挙げられる. word2vecを用いた文書検索法は,TF-IDFの検索精度が高くないクエリにおいて大きく精度を向上させる傾向があったため,
% 互いの欠点を補い合うような組み合わせ方をすることで,精度向上に繋がるのではないかと考えられる.

% 音声認識誤りの棄却において,棄却の閾値を変化させた場合にどのように精度が変化するのかを確認しなくてはならない.
% 検索に有用な語を棄却してしまったため,閾値を負に設定することで,棄却の厳しさを緩和させることができる.
% また$sumPMI$が負になりやすい語に,何かしらの傾向が見られないか調査することで,検索に有用な語を棄却する問題を解決できると期待できる.

% 最後に,本研究で取り上げた手法を複数組み合わせることで,検索精度がどのように変化するか確認することが重要である.
% 本研究では要素を1つ1つ取り上げ,それぞれを独立に実験した.
% 一方でこれらの手法は組み合わせることができる.これらの手法を組み合わせることで,更なる精度の向上が期待できる.
